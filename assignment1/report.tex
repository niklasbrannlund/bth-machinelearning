\documentclass{article}
\usepackage[utf8]{inputenc}
\usepackage{amsmath}
\usepackage{flexisym}
\title{DV2574 - Assignment 1}
\author{Niklas Brännlund}
\date{February 2019}

\begin{document}
\maketitle

\section{Problems}
\subsection{Correlation coefficients}
    Given the sample data
    \begin{equation}
        \begin{aligned}
            X_{1} & = 
            \begin{bmatrix}
               2.5 & 3.6 & 1.2 & 0.8 & 4.0 & 3.4
            \end{bmatrix} \\
            X_{2} & = 
            \begin{bmatrix}
               1.2 & 1.0 & 1.8 & 0.9 & 3.0 & 2.2
            \end{bmatrix}\\
            X_{3} & = 
            \begin{bmatrix}
           8.0 & 15.0 & 12.0 & 6.0 & 8.0 & 10.0
            \end{bmatrix}
        \end{aligned}
    \end{equation}
    the correlation coefficient matrix was calculated to be \\
    \begin{equation}
        \begin{bmatrix}
            1.00 & 0.53 & 0.31\\
            0.53 & 1.0000 & -0.13\\
            0.31 & -0.13 & 1.00\\
        \end{bmatrix}
    \end{equation}

    From this matrix we can extract the correlation coefficients $\rho_{X_iX_j}$ to be  
    \begin{equation}
        \begin{aligned}
            \rho_{X_1X_2} & = 0.53\\
            \rho_{X_1X_3} & = 0.31\\
            \rho_{X_2X_3} & = -0.13\\
        \end{aligned}
    \end{equation}
    This was done by calculating the quantity 
    \begin{equation}
        \rho_{m,n} = \frac{S_{X_mX_n}}{S_{X_{m}X_{m}}S_{X_{n}X_{n}}}
    \end{equation}
    where $m,n \in \{1, 2, 3\}$. The quantity $S_{X_{m}X_{n}}$ is the covariance and is calculated as 
    \begin{equation}
        S_{X_{m}X_{n}} = \sum_{i=1}^{k} \frac{(x_{m_i} - \bar{X}_m)(x_{n_i} - \bar{X}_N)}{k-1}
    \end{equation}
    and $S_{X_mX_m}$, which is the standard deviation of dataset $X_m$, is calculated as 
    \begin{equation}
        S_{X_{m}X_{m}} = \sqrt{\sum_{i=1}^{k} \frac{(x_{m_i} - \bar{X}_{m_i})^2}{n-1}}.
        \label{eq:std}
    \end{equation}
    The quantity $S_{X_nX_n}$ is calculated in the same way as \eqref{eq:std}. This result was also compared against matlabs own           implementation corrcoef\footnote{https://se.mathworks.com/help/matlab/ref/corrcoef.html}
 
    \subsection{K nearest neighbor}
    Given the following dataset of house prices together with house variables 
    \begin{equation}
        \begin{tabular}{ c | c | c | c }
            y: Price & $x_1$: Rooms & $x_2$: Size ($m^2$) & $x_3$: Age of house (years)\\ \hline
            500000 & 2 & 45 & 25\\
            800000 & 3 & 65 & 30\\ 
            1000000 & 6 & 100 & 40\\
            350000 &2 & 30 & 20 \\
            100000 & 2 & 25 & 20 
        \end{tabular}
    \end{equation}
    The assignment is to use k-nearest-neighbor method to estimate house prices with the following properties
    \begin{equation}
        \begin{tabular}{ c | c | c | c }
            y: Price & $x\textprime_1$: Rooms & $x\textprime_2$: Size ($m^2$) & $x\textprime_3$: Age of house (years)\\ \hline
            ? & 4 & 100 & 25\\
            ? & 1 & 60 & 20\\ 
        \end{tabular}
    \end{equation}
    using both k = 1 and k = 2. This was done by calculating the manhattan distance D between each datapoint using following function
    \begin{equation}
        D = \sum_{i=1}^n |x_i - x\textprime_i|
    \end{equation}
    This will give you a distance vector which can be sorted and from there you can retrieve the $k$-closest neighbors by choosing the first $k$ elements in this vector. From these values you can pick out the house price values that these neighbors had from the training dataset. If $k > 1$, the mean value of these house prices were calculated to retrieve the estimation, see implementation in code. This approach resulted in the following house price estimations
    
    \begin{equation}
        \begin{tabular}{ c | c | c | c }
            y: Price & $x\textprime_1$: Rooms & $x\textprime_2$: Size ($m^2$) & $x\textprime_3$: Age of house (years)\\ \hline
            1000000 & 4 & 100 & 25\\
            800000 & 1 & 60 & 20\\ 
        \end{tabular}
    \end{equation}
    for k = 1, and for k = 2 the result was the following
    \begin{equation}
        \begin{tabular}{ c | c | c | c }
            y: Price & $x\textprime_1$: Rooms & $x\textprime_2$: Size ($m^2$) & $x\textprime_3$: Age of house (years)\\ \hline
            900000 & 4 & 100 & 25\\
            650000 & 1 & 60 & 20\\ 
        \end{tabular}
    \end{equation}
    
\end{document}
