\documentclass{article}
\usepackage[utf8]{inputenc}
\usepackage{amsmath}
\usepackage{flexisym}
\usepackage[]{algorithm2e}
\usepackage{url}
\usepackage{enumitem}
\usepackage{listings}
\title{DV2574 - Assignment 2}
\author{Niklas Brännlund}
\date{March 2019}

\begin{document}
\maketitle

\section*{Problems}
    \section{Neural networks}
    This part is about constructing a two-input perceptron neural network model and using an algorithm for finding the final weights of the model. The inputs  $x_1$ and $x_2$ is described using the logical \textbf{OR} operator
    \begin{equation}
        \begin{tabular}{ | c | c | c | }
            \hline
            \multicolumn{2}{ |c| }{\textbf{Input variables}} & \textbf{OR} \\ 
            $\boldsymbol{x_1}$ & $\boldsymbol{x_2}$ & $\boldsymbol{x_1}$ $\cup$ $\boldsymbol{x_2}$ \\ \hline
            0 & 0 & 0\\
            0 & 1 & 1\\
            1 & 0 & 1\\
            1 & 1 & 1\\ \hline
        \end{tabular}
        \label{eq:or}
    \end{equation}
    Using the input from eq. \eqref{eq:or} in the following algorithm \\

    \begin{enumerate}
        \item Set initial weights $(w_1, w_2) = (0.5, 0.1)$ and bias $\theta = -0.1$. Learning rate $\alpha$ should be set to $\alpha = 0.01$ 
        \item Activate the model by applying inputs $x_1(p), x_2(p)$ and the output from the OR operator $Y_d(p)$ (which is the desired output). Calculate the actual output for iteration p=1 
        \begin{equation}
            Y(p) = step\Bigg[\sum_{i=1}^n x_i(p)w_i(p) - \theta\Bigg]
        \end{equation}
        where $step$ is a step function.
        \item Update the weights with the following equation
        \begin{equation}
            w_i(p+1) = w_i(p) + \Delta w_i(p)
        \end{equation}
        where $\Delta w_i(p)$ is the weight correction. It is defined as 
        \begin{equation}
            \Delta w_i(p) = \alpha\ x_i(p) e(p)
        \end{equation}
        where $e(p)$ is the difference between the desired output and the calculated result
        \begin{equation}
            e(p) = Y_d(p) - Y(p)
        \end{equation}
    \end{enumerate}
    By running the script \texttt{perceptron-neural-network.py} we get the output for the final weights to be
    $w_1=0.1$ and $w_2 = 0.5$.
    The algorithm logic for perceptron neural network can be found under \url{/algorithm/neuralnetwork.py}.
    \section{K means clustering}
    This section handles problem where k-means-algorithm is applied.
        \subsection{Image comparison}
        \label{im}
        Given two images \textbf{imA} and \textbf{imB} where each image is presented by by a 10x10 pixel matrix. Find out for which pixel coordinated there is a big difference in pixel values. This is done by first calculating a difference image \textbf{im}
        \begin{equation}
            \textbf{im} = |\textbf{imA} - \textbf{imB}|
        \end{equation}
        Given the difference image, we normalize this image with the following equation
        \begin{equation}
            \textbf{imNormalized} = \frac{\textbf{im} - im_{min}}{im_{max} - im_{min}}
        \end{equation}
        where $im_{min}$ and $im_{max}$ is the min and max value of \textbf{im}, respectively. Now we can apply K-means-clustering on \textbf{imNormalized} to obtain which pixels that have changed between \textbf{imA} and \textbf{imB}. The result is printed as a result matrix 
        
        \begin{equation}
            \textbf{imResult} = 
            \begin{bmatrix}
                0 & 0 & 0 & 0 & 0 & 0 & 0 & 0 & 0 & 0 \\
                0 & 0 & 0 & 0 & 0 & 0 & 0 & 0 & 0 & 0 \\
                0 & 0 & 0 & 0 & 0 & 0 & 0 & 0 & 0 & 0 \\
                0 & 0 & 0 & 0 & 0 & 0 & 0 & 0 & 0 & 0 \\
                0 & 0 & 0 & 0 & 0 & 0 & 0 & 0 & 1 & 1 \\
                0 & 0 & 0 & 0 & 0 & 0 & 0 & 0 & 1 & 1 \\
                0 & 0 & 0 & 0 & 0 & 0 & 0 & 0 & 1 & 1 \\
                1 & 1 & 1 & 1 & 1 & 1 & 1 & 1 & 1 & 1 \\
                1 & 1 & 1 & 1 & 1 & 1 & 1 & 1 & 1 & 1 \\
                1 & 1 & 1 & 1 & 1 & 1 & 1 & 1 & 1 & 1
            \end{bmatrix}
        \end{equation}
        where 1 stands for a significant change in pixel value and 0 is no significant change. This result can be reproduced by running the script \texttt{image-comparison.py}. The algorithm logic for k-means-clustering can be found under \url{/algorithm/kmeansclustering.py}.
        
        \subsection{Corrupted currency}
        Given a list of crypto currency prices that were gathered during a day. Use k-means clustering to find prices that were corrupted when they were stored. This problem uses the same algorithm as in section \ref{im}. Run the script \texttt{corrupted-currency.py} to get the result printed out. When using 6 clusters (setting NUM\_CLUSTERS = 6 in script) we get the following result
        
        \begin{enumerate}[label=(\alph*)]
            \item The corrupted prices were found to be 
            \begin{enumerate}
                \item 100587\$
                \item 101964\$
                \item 120967\$
            \end{enumerate}
            \item The max prices were found to be
            \begin{enumerate}
                \item 7845.0\$
                \item 7956.0\$
                \item 8215.0\$
                \item 8542.0\$
                \item 8150.0\$
                \item 8386.0\$
                \item 8219.0\$
                \item 7500.0\$
                \item 9257.0\$
                \item 8553.0\$
            \end{enumerate}
            \item the minimum prices were found to be
            \begin{enumerate}
	            \item 0.76\$
            	\item 0.78\$
            	\item 0.78\$
            	\item 0.72\$
            	\item 0.82\$
            	\item 0.82\$
            	\item 0.67\$
            	\item 0.72\$
            	\item 0.84\$
            \end{enumerate}
        \end{enumerate}
        We can see that one cluster is created where the values are above the max price of 20.000\$ and these are the prices that are to be seen as corrupt. When we use for example 4 clusters, we can still find the corrupted prices as they create an identical cluster as in the 6-cluster-case. Although, the min prices-cluster is different compared to when using 6 clusters. This can be seen when running the script by setting the NUM\_CLUSTERS variable to 4.
\end{document}
